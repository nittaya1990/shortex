\documentclass{article}

\usepackage{shortex}

\title{The \texttt{ShorTeX} package}
\author{Trevor Campbell, Jonathan Huggins, and Jeff Negrea}
\date{Updated \today}


\begin{document}

\maketitle


\babs
The purpose of the ShorTeX (meta)package is to make the process of typesetting
typical mathematical documents in \LaTeX~more efficient, and the resulting
code easier to read.  It achieves this by 
(1) providing an
extensive, internally consistent, and easy to learn set of macro
shorthands and custom commands, and 
(2) incorporating a set of packages that are
dedicated to reducing manual coding effort.
\eabs


\tableofcontents

\section{Usage and package options}\label{sec:usage}


Include ShorTeX by adding \verb!\usepackage{shortex}! to the preamble of your document.
ShorTeX will include and configure many common packages for you (e.g., \texttt{graphicx, subcaption, hyperref, algorithm, algpseudocode, amsmath}, among others),
so you do not need to explicitly include and set these up yourself.
If you are writing a document that must use a specific style file (e.g., for a conference or journal) that itself
includes some of these packages, we recommend editing those style files to remove the package imports.

The ShorTeX package has a few options:
\bdesc
\item[\texttt{manualnumbering}] Do not include \texttt{autonum.sty}. This disables automatic equation numbering.
\item[\texttt{blackhypersetup}] Switch hyperlinks, citations, references, etc.~to be typeset in black font. The default is dark blue.
\edesc
\textbf{You must compile your document 4 times when using ShorTeX} to ensure that equation
numbers and references update properly.

\section{Packages included in ShorTeX}
\subsection{Internal packages}
The following packages provide features that are used internally in ShorTeX for 
macro definitions etc.
\bdesc
\item[xifthen, xstring, xspace, xargs] asdf
\edesc
\subsection{Typical packages}
We include a set of typical packages to avoid the problem of having to remember precisely
what order to import each one, which options, compatibility, etc

It is worth noting that we did not include a standard bibliography package in ShorTeX (e.g., \verb!natbib.sty!).
This is because...

\bdesc
\item[mathrsfs,dsfont,amsmath,amssymb,amsthm,bm,bbm,amsfonts,mathtools,thmtools] asdf
\item[hyperref] asdf
\item[color] asdf
\item[algorithm, algpseudocode] asdf
\item[graphicx] asdf
\edesc
\subsection{Improvement packages}
\bdesc
\item[cleveref] Typically to use a reference in \LaTeX, you have to write the name of the type of reference
yourself. For example, if you want to reference a figure, you would have to write something like:
\begin{verbatim}
In Figure \ref{fig:first}, you can see...
\end{verbatim}
Or for multiple figures, you might use:
\begin{verbatim}
Figures \ref{fig:first}, \ref{fig:second}, 
and \ref{fig:third} show that...
\end{verbatim}

The \texttt{cleveref} package simplifies this process significantly. Use the \verb!\cref! command to automatically
typeset the names of the objects you're referencing (including properly handling multiple references). The 
above two examples become
\begin{verbatim}
In \cref{fig:first}, you can see...
\end{verbatim}
and
\begin{verbatim}
\cref{fig:first,fig:second,fig:third} show that...
\end{verbatim}

This works for many different reference types (Figure, Algorithm, Equation, Table, etc),
and can be extended if needed. See the \texttt{cleveref} documentation 
at \url{https://ctan.org/pkg/cleveref?lang=en} and the homepage at \url{https://www.dr-qubit.org/cleveref.html} 
for more information.

\item[autonum] 
Typically when you typeset equations, you have to choose between 
\verb!$...$!, \verb!$$...$$!, 
\verb!\begin{align}...\end{align}!, 
\verb!\begin{aligned}...\end{aligned}!, 
\verb!\begin{equation}...\end{equation}!, 
not to mention starred versions of those environments 
and \verb!\nonumber!/\verb!\notag! commands, depending 
on whether/where you want equation numbers,
display or in-text math, etc. This leads to verbose, inconsistent code.

The \texttt{autonum} package automatically decides which equations to provide
numbers based on \textit{which equations you reference}. So when using ShorTeX,
you only need two commands for math mode: single dollar signs \verb!$...$! for
inline math, and \texttt{align} environments (redefined in ShorTeX to be
\verb!\[...\]!) for display math.\footnote{Note that there are minor
differences between how \texttt{align} and 
\texttt{equation} display equations. But after 10+ years of using \LaTeX, I have not ever
encountered a case where it mattered much. That being said, ShorTeX does not
\emph{disable} any functionality, so you can use the usual environments anywhere
you feel it is necessary.}

For example, if you create 
the following display math,
\begin{verbatim}
\[
   a+b = c \label{eq:the_equation}
\]
\end{verbatim}
then if you use the command \verb!\cref{eq:the_equation}! somewhere
in the document, that equation will automatically be assigned a number. If not, it
won't get a number. See the \texttt{autonum} package 
documentation \url{https://ctan.org/pkg/autonum?lang=en} for more information.

\item[nicefrac] asdf
\item[crossreftools] asdf
\item[multirow] asdf
\item[wrapfig] asdf
\item[caption,subcaption] asdf
\item[microtype] asdf
\item[booktabs] asdf
\item[import,subfiles] asdf
\item[url] asdf
\edesc

\section{Shorthands for existing commands}

\subsection{Environments}

\LaTeX~documents often includes a lot of verbose code
related to creating environments (\verb!\begin{blah}...\end{blah}!). ShorTeX provides a set of 
shortened macros for common environments.
Note that all theorem-like environments (theorem, lemma, proposition, etc.) 
are numbered by default; unnumbered versions can be obtained by appending a \verb!u!. For example,
\verb!\bthmu...\ethmu! creates an unnumbered theorem environment, while
\verb!\blemu...\elemu! creates an unnumbered lemma environment.

\bcent
\btabr{@{}ll@{}}
\toprule
Environment & Syntax \\ \midrule
abstract & \verb!\babs...\eabs!\\ \midrule
itemize & \verb!\bitems...\eitems!\\
enumerate & \verb!\benum...\eenum!\\
description & \verb!\bdesc...\edesc!\\ \midrule
algorithm & \verb!\balg...\ealg!\\
algorithmic & \verb!\balgc...\ealgc!\\ \midrule
table & \verb!\btab...\etab!\\
subtable & \verb!\bsubtab...\esubtab!\\
tabular & \verb!\btabr...\etabr!\\ \midrule
figure & \verb!\bfig...\efig!\\
figure* & \verb!\bfigs...\efigs!\\
subfigure & \verb!\bsubfig...\esubfig!\\ \midrule
center & \verb!\bcent...\ecent!\\ \midrule
align & \verb!\[...\]!\\ 
inline math & \verb!$...$!\\ \midrule
\multicolumn{2}{c}{\emph{Note: These are numbered theorem-like environments.}}\\
\multicolumn{2}{c}{\emph{For unnumbered, append a \texttt{u}: e.g.,} \texttt{bthmu...ethmu}.}\\
theorem & \verb!\bthm...\ethm!\\ 
lemma & \verb!\blem...\elem!\\
proposition & \verb!\bprop...\eprop!\\
corollary & \verb!\bcor...\ecor!\\
conjecture & \verb!\bconj...\econj!\\
definition & \verb!\bdef...\edef!\\
assumption & \verb!\bassump...\eassump!\\
example & \verb!\bexa...\eexa!\\
remark & \verb!\brmk...\ermk!\\
fact & \verb!\bfact...\efact!\\
exercise & \verb!\bexer...\eexer!\\
proof & \verb!\bprf...\eprf!\\
proofof & \verb!\bprfof{\cref{theorem_label}}...\eprfof!\\  \midrule
matrix & \verb!\bmat...\emat!\\
bmatrix & \verb!\bbmat...\ebmat!\\
pmatrix & \verb!\pbmat...\epmat!\\
\bottomrule
\etabr
\ecent

\subsection{Delimiters}

Mathematics in \LaTeX~often includes quite a few delimiters (parentheses, brackets, curly brackets, etc.).
A very common usage of these involves the \verb!\left...\right! commands for automatic sizing. 
One can also use \verb!\bigl...\bigr!, \verb!\Bigl...\Bigr!, \verb!\biggl...\biggr!, \verb!\Biggl...\Biggr! to control sizing manually.
ShorTeX creates shorthands for these.

\bcent
\btabr{@{}llll@{}}
\toprule
Description & Syntax  \\ \midrule
automatic	& \verb!\lt...\rt!\\        
big 	& \verb!\lb...\rb!\\
Big & \verb!\lB...\rB! \\ 
bigg & \verb!\lbg...\rbg!\\ 
Bigg & \verb!\lBg...\rBg!\\
\bottomrule
\etabr
\ecent

These can be applied to all the usual delimiter characters.
The following tables demonstrate usage for automatically sized delimiters. 

\bcent
\btabr{@{}llll@{}}
\toprule
Description & Example & Text style & Display style \\ \midrule
parentheses	& \verb!\lt(\frac{x}{y}\rt)!        	& $\lt(\frac{x}{y}\rt)$ 		& $\displaystyle\lt(\frac{x}{y}\rt)$ \\[10pt]
curly brackets 	& \verb!\lt\{\frac{x}{y}\rt\}!    	& $\lt\{\frac{x}{y}\rt\}$ 	& $\displaystyle\lt\{\frac{x}{y}\rt\}$ \\[10pt]
square brackets & \verb!\lt[frac{x}{y}\rt]!        	& $\lt[\frac{x}{y}\rt]$ 	& $\displaystyle\lt[\frac{x}{y}\rt]$ \\[10pt]
pipes & \verb!\lt|frac{x}{y}\rt|!        	& $\lt|\frac{x}{y}\rt|$ 	& $\displaystyle\lt|\frac{x}{y}\rt|$ \\[10pt]
double pipes & \verb!\lt\|frac{x}{y}\rt\|!        	& $\lt\|\frac{x}{y}\rt\|$ 	& $\displaystyle\lt\|\frac{x}{y}\rt\|$ \\[10pt]
angle brackets & \verb!\lt<frac{x}{y}\rt>!        	& $\lt<\frac{x}{y}\rt>$ 	& $\displaystyle\lt<\frac{x}{y}\rt>$ \\[10pt]
\bottomrule
\etabr
\ecent

\subsection{Font styles and accents}

Applying accents (e.g., hats $\f[h]a$, tildes $\f[t]a$, bars $\f[b]a$)
and changing fonts (e.g., doublestroke $\f[d]A$, caligraphic $\f[c]A$, and bold $\f[k]A$)
is quite cumbersome in standard \LaTeX. For example, the code to make a tilde caligraphic A,
$\widetilde{\mathcal{A}}$
is \verb!\widetilde{\mathcal{A}}!. By itself that code is not too bad, but many such characters 
in a large mathematical expression results in unreadable code.

ShorTeX defines an efficient syntax for changing fonts and applying accents to characters. 
The syntax takes the form \verb!\f[modifiers]character!, where \verb!modifiers! is a set of single characters
that represent font/accent modifications to \verb!character!. 
For example, the code for tilde caligraphic A is \verb!\f[tc]A! where \verb!t! represents ``tilde,'' \verb!c! represents
``caligraphic,'' and \verb!A! is the character to typeset.

\emph{Note: modifiers are applied in the reverse of the order in which they appear; 
the modifier furthest to the right is applied first. This matches the order that 
the corresponding commands would appear in TeX code.}

\bcent
\btabr{@{}llll@{}}
\toprule
Style/Accent & Modifier & Example & Typeset Example \\ \midrule
caligraphic (\verb!mathcal!) & \verb!c! & \verb!\f[c]A! & $\f[c]A$ \\
bold (\verb!mathbf!) & \verb!k! & \verb!\f[k]A! & $\f[k]A$\\
doublestroke (\verb!mathbb!) & \verb!d! & \verb!\f[d]A! & $\f[d]A$\\
hat (\verb!widehat!) & \verb!h! & \verb!\f[h]A! & $\f[h]A$\\
tilde (\verb!widetilde!) & \verb!t! & \verb!\f[t]A! & $\f[t]A$\\
bar (\verb!widebar!) & \verb!b! & \verb!\f[b]A! & $\f[b]A$\\
\bottomrule
\etabr
\ecent

These style modifiers can be combined; the underlying code is flexible enough that
it will happily produce a wide variety of combinations, including those that aren't very sensible.

\bcent
\btabr{@{}llll@{}}
\toprule
Style/Accent & Modifier & Example & Typeset Example \\ \midrule
caligraphic tilde & \verb!ct! & \verb!\f[ct]A! & $\f[ct]A$ \\
bold hat & \verb!kh! & \verb!\f[kh]A! & $\f[kh]A$\\
hat tilde  & \verb!ht! & \verb!\f[ht]A! & $\f[ht]A$\\
tilde hat  & \verb!th! & \verb!\f[th]A! & $\f[th]A$\\
\bottomrule
\etabr
\ecent

We can avoid typing \texttt{[]} for commonly used patterns
by parsing the font style string in advance.
For example, if we use ``bold hat'' symbols frequently,
we might want to use commands like
\verb!\fkh...!  instead of \verb!\f[kh]...!.
We can accomplish this using the \verb!\parsefontstylestrings! command,
with syntax
\begin{verbatim}
\parsefontstylestrings{{<fstr1>}{<fstr2}...}{<alphabet>}
\end{verbatim}
For example, to define ``bold hat'' and ``caligraphic hat'' styles
for the characters A, B, C, and D, we would use the command 
\begin{verbatim}
\parsefontstylestrings{{kh}{ch}}{ABCD}
\end{verbatim}
\parsefontstylestrings{{kh}{ch}}{ABCD}
and then in the \LaTeX~document, use the commands
\verb!\fkhA \fkhB \fkhC \fkhD! and
\verb!\fchA \fchB \fchC \fchD! 
to obtain the following symbols:
\[
\fkhA \fkhB \fkhC \fkhD 
\fchA \fchB \fchC \fchD 
\]
As another example, for ``bold hat'' applied to $\alpha$, $\beta$, and $\gamma$, we would use the syntax
\begin{verbatim}
\parsefontstylestrings{{kh}}{\alpha\beta\gamma}
\end{verbatim}
\parsefontstylestrings{{kh}}{\alpha\beta\gamma}
and then in the \LaTeX~document, use the commands
\verb!\fkhalpha \fkhbeta \fkhgamma!
to obtain the following symbols:
\[
	\fkhalpha \fkhbeta \fkhgamma
\]

For convenience we also provide a few common alphabets of symbols 
for use in the \verb!\parsefontstylestrings! command.
Note that not every Greek character has an uppercase version (in cases where it is
identical to its Roman counterpart).

\bcent
\btabr{@{}lll@{}}
\toprule
Syntax & Characters  \\ \midrule
\verb!\lowercaseRoman! & abcdefghijklmnopqrstuvwxyz \\
\verb!\uppercaseRoman! & ABCDEFGHIJKLMNOPQRSTUVWXYZ \\
\verb!\lowercaseGreek! & alpha,beta,gamma,delta,epsilon,zeta,eta,theta\\
& iota,kappa,lambda,mu,nu,xi,omicron,pi,rho\\
& sigma,tau,upsilon,phi,chi,psi,omega\\
\verb!\uppercaseGreek! & Gamma,Delta,Theta,Lambda,Xi,Pi,Sigma\\
& Upsilon,Phi,Psi,Omega \\
\bottomrule
\etabr
\ecent

\subsection{Greek characters and variants}

ShorTeX defines a number of shorthands for Greek characters and variants.

\bcent
\btabr{@{}lll@{}}
\toprule
Letter & Syntax & Symbol  \\ \midrule
epsilon	& \verb!\eps! & $\eps$ \\
upsilon	& \verb!\ups! & $\ups$ \\
variant epsilon	& \verb!\veps! & $\veps$ \\
variant theta	& \verb!\vtheta! & $\vtheta$ \\
variant pi	& \verb!\vpi! & $\vpi$ \\
variant rho	& \verb!\vrho! & $\vrho$ \\
variant sigma	& \verb!\vsigma! & $\vsigma$ \\
variant phi	& \verb!\vphi! & $\vphi$ \\
variant kappa	& \verb!\vkappa! & $\vkappa$ \\
\bottomrule
\etabr
\ecent


\section{Custom macros}

\subsection{Shrinking whitespace in math}
The command \verb!\squish{<frac>}! in math mode enables you to shrink whitespace in mathematics,
where \verb!<frac>! represents the fraction of whitespace reduction.
Below, the first line is regularly spaced, the second line has \verb!\squish{0.5}!, and the third has \verb!\squish{0.0}!.
\[
	\sqrt{\frac{1^{2}}{0.111222}(0.111222\times1.111163+0.066987^{2}\times0.111222)-1}&= \sqrt{0.111222}\\
	\squish{0.5}\sqrt{\frac{1^{2}}{0.111222}(0.111222\times1.111163+0.066987^{2}\times0.111222)-1}&= \sqrt{0.111222}\\
	\squish{0.0}\sqrt{\frac{1^{2}}{0.111222}(0.111222\times1.111163+0.066987^{2}\times0.111222)-1}&= \sqrt{0.111222}\\
\]

The code for \verb!\squish! was taken from \url{https://tex.stackexchange.com/questions/467942/how-to-squeeze-a-long-equation}.

\subsection{Wide bar}

ShorTeX provides the \verb!\widebar! command to typeset a wide bar accent on top of a character (similar to the 
usual \verb!\widehat! and \verb!\widetilde! commands). Compare to the usual \verb!\bar! and 
\verb!\overline! commands:
\[
	\text{\texttt{widebar}:}\,\, \widebar{A} \qquad \text{\texttt{overline}:} \,\,\overline{A} \qquad \text{\texttt{bar}:} \,\,\bar{A}
\]
The code for \verb!\widebar! was taken from \url{https://tex.stackexchange.com/questions/16337/can-i-get-a-widebar-without-using-the-mathabx-package}.



\subsection{Commenting}
ShorTeX defines two types of comments that can be used 
(\emph{remarks} and \emph{problems}), and provides an inline and margin
style for each.

\bcent
\btabr{@{}ll@{}}
\toprule
Comment Type & Syntax \\ \midrule
remark & \verb!\RMK{Example remark}!\\ 
margin remark & \verb!\mRMK{Example margin remark}!\\ 
problem & \verb!\PRB{Example problem}!\\ 
margin problem & \verb!\mPRB{Example margin problem}!\\ 
\bottomrule
\etabr
\ecent

Here is an example of how these look in a typical paragraph:\\
\\

Lorem ipsum dolor sit amet \RMK{Here is an inline remark}, consectetur adipiscing elit, sed do eiusmod tempor
incididunt ut labore et dolore magna aliqua. Ut enim ad minim veniam, quis
nostrud exercitation \PRB{Here is an inline problem} ullamco laboris nisi ut aliquip ex ea commodo consequat.
Duis aute irure dolor in reprehenderit in \mRMK{Here is a margin remark} voluptate velit esse cillum dolore eu
fugiat nulla pariatur. Excepteur sint occaecat \mPRB{Here is a margin problem} cupidatat non proident, sunt...




\subsection{Sets and set operations}

\[
\reals \extReals \posReals \posExtReals \posPosReals \posPosExtReals
\ints \posInts \rats \posRats \nats \natsO \comps \measures
\probMeasures \PowerSet \union \Union \djunion \djUnion
\intersect \Intersect \vol \diam \closure \spann \boundary \cone \conv
\]

\subsection{Linear algebra}

\[
\tr \kron A\adj \spec \diag \rank A\transpose A\invtranspose
\]

\subsection{Calculus}

\[
\dee \grad \der{x}{y} \dder{x}{y} \derwrt{y} \pder{x}{y} \pdder{x}{y} \pderi{x}{y} \pderwrt{y} \hes{a}{x}{y} \heswrt{x}{y}
\]

\subsection{General mathematics}

\[
\ind \sgn \scin{3}{5} \st \given
\] 

improved square root

text in math?

\[
\defas \defines \half \third \quarter \argmax \esssup \argmin \essinf
\]


\subsection{Common words and names with accents}
\cadlag
\Gronwall
\Renyi
\Holder
\Ito
\Nystrom
\Schatten
\Matern
\Frechet
\Levy

\subsection{Probability and statistics}
\iid \as \aev 
\[
\convas \convp \convd \eqd \eqas \Pr \EE \law \var \cov \corr \supp \dist \distiid \distind
\indep
\]

\[
\distNorm \distT \distWish \distInvWish \distLap \distChiSq \distUnif \distGam \distGumbel \distGEV \distCat \distInvGam \distPoiss \\
\distExp \distBeta \distBetaPrime \distDir \distBinom \distMulti \distBern \distGeom \distCauchy \distVMF \\
\distBeP \distDP \distCRP \distPYP \distGP \distPP \distBP \distBPP \distGamP \distNGamP \distLP \distObs \distCRM
\distNCRM
\]

\[
\kl{q}{p} \hell{q}{p} \tvd{q}{p} \ent{q}\\
\hell[a]{q}{p}
\tvd[a]{q}{p}
\kl[a]{q}{p}
\]

\subsection{$\Lp{p}$ Spaces and Operators}

\begin{center}
\begin{tabular}{@{}llll@{}}
\toprule
Description						& Example 				& Text style 				& Display style \\ \midrule
$\Lp{p}$ space					& \verb!\Lp{2}!        		& $\Lp{2}$ 				& $\displaystyle\Lp{2}$ \\[10pt]
\begin{tabular}[c]{@{}l@{}}$\Lp{p}$ space for \\ specified measure	 \end{tabular}	& \verb!\Lpmeas{2}{\f[h]\mu}!	& $\Lpmeas{2}{\f[h]\mu}$ 		& $\displaystyle\Lpmeas{2}{\f[h]\mu}$ \\[10pt]
							& \verb!\Lpmeas[\Big]{2}{\f[h]\mu}!	& $\Lpmeas[\Big]{2}{\f[h]\mu}$ 	& $\displaystyle\Lpmeas[\Big]{2}{\f[h]\mu}$ \\[10pt]
$\Lp{p}$ norm					& \verb!\Lpnorm{\f[h]\Gamma}{2}!        & $\Lpnorm{\f[h]\Gamma}{2}$ 		& $\displaystyle\Lpnorm{\f[h]\Gamma}{2}$ \\[10pt]
							& \verb!\Lpnorm*{\f[h]\Gamma}{2}!        & $\Lpnorm*{\f[h]\Gamma}{2}$ 		& $\displaystyle\Lpnorm*{\f[h]\Gamma}{2}$ \\[10pt]
				& \verb!\Lpnorm*{\Gamma}{2}!        & $\Lpnorm*{\Gamma}{2}$ 		& $\displaystyle\Lpnorm*{\Gamma}{2}$ \\[10pt]
							& \verb!\left\Vert{\f[h]\Gamma}\right\Vert_{2}!        & $\left\Vert{\f[h]\Gamma}\right\Vert_{2}$ 		& $\displaystyle\left\Vert{\f[h]\Gamma}\right\Vert_{2}$ \\[10pt]
							& \verb!\left\Vert{\Gamma}\right\Vert_{2}!        & $\left\Vert{\Gamma}\right\Vert_{2}$ 		& $\displaystyle\left\Vert{\Gamma}\right\Vert_{2}$ \\[10pt]														
\begin{tabular}[c]{@{}l@{}}$\Lp{p}$ norm for \\ specified measure	 \end{tabular}		& \verb!\Lpmeasnorm{\f[h]\Gamma}{2}{\f[h]\mu}!        & $\Lpmeasnorm{\f[h]\Gamma}{2}{\f[h]\mu}$ 		& $\displaystyle\Lpmeasnorm{\f[h]\Gamma}{2}{\f[h]\mu}$ \\[10pt]
							& \verb!\Lpmeasnorm[\Big]{\f[h]\Gamma}{2}{\f[h]\mu}!        & $\Lpmeasnorm[\Big]{\f[h]\Gamma}{2}{\f[h]\mu}$ 		& $\displaystyle\Lpmeasnorm[\Big]{\f[h]\Gamma}{2}{\f[h]\mu}$ \\[10pt]
$\Lp{p}$ inner product				& \verb!\Lpinner{\f[h]\Gamma}{\Gamma}{2}!        & $\Lpinner{\f[h]\Gamma}{\Gamma}{2}$ 		& $\displaystyle\Lpinner{\f[h]\Gamma}{\Gamma}{2}$ \\[10pt]
							& \verb!\Lpinner*{\f[h]\Gamma}{\Gamma}{2}!        & $\Lpinner*{\f[h]\Gamma}{\Gamma}{2}$ 		& $\displaystyle\Lpinner*{\f[h]\Gamma}{\Gamma}{2}$ \\[10pt]
\begin{tabular}[c]{@{}l@{}}$\Lp{p}$ inner product \\  for specified measure	 \end{tabular} & \verb!\Lpmeasinner{\f[h]\Gamma}{\Gamma}{2}{\f[h]\mu}!        & $\Lpmeasinner{\f[h]\Gamma}{\Gamma}{2}{\f[h]\mu}$ 		& $\displaystyle\Lpmeasinner{\f[h]\Gamma}{\Gamma}{2}{\f[h]\mu}$ \\[10pt]
							& \verb!\Lpmeasinner[\big]{\f[h]\Gamma}{\Gamma}{2}{\f[h]\mu}!        & $\Lpmeasinner[\big]{\f[h]\Gamma}{\Gamma}{2}{\f[h]\mu}$ 		& $\displaystyle\Lpmeasinner[\big]{\f[h]\Gamma}{\Gamma}{2}{\f[h]\mu}$ \\[10pt]
\bottomrule
\end{tabular}
\end{center}


\subsection{Paired Delimiters}

You can specify a bracket size using $*$ for \verb!\left! and \verb!\right! or one of the standard size choices (\verb!\big!, \verb!\Big!, \verb!\bigg!, \verb!\Bigg!).

\begin{center}
\begin{tabular}{@{}llll@{}}
\toprule
Description 				& Example					& Text style 				& Display style \\ \midrule
Round brackets	& \verb!\rbra{\frac{x}{y}}!        	& $\rbra{\frac{x}{y}}$ 		& $\displaystyle\rbra{\frac{x}{y}}$ \\[10pt]
Curly brackets 			& \verb!\cbra*{\frac{x}{y}}!    	& $\cbra*{\frac{x}{y}}$ 	& $\displaystyle\cbra*{\frac{x}{y}}$ \\[10pt]
Square brackets 			& \verb!\sbra[\bigg]{\frac{x}{y}}!        	& $\sbra[\bigg]{\frac{x}{y}}$ 	& $\displaystyle\sbra[\bigg]{\frac{x}{y}}$ \\[10pt]
\bottomrule
\end{tabular}
\end{center}

Many other bracket-like, semantic commands are also available:

\begin{center}
\begin{tabular}{@{}llll@{}}
\toprule
Description				& Example 					& Text style 				& Display style \\ \midrule
%\verb!\rbra{\frac{x}{y}}!        	& $\rbra{\frac{x}{y}}$ 		& $\displaystyle\rbra{\frac{x}{y}}$ \\[10pt]
%\verb!\cbra{\frac{x}{y}}!        	& $\cbra{\frac{x}{y}}$ 		& $\displaystyle\cbra{\frac{x}{y}}$ \\[10pt]
%\verb!\sbra{\frac{x}{y}}!        	& $\sbra{\frac{x}{y}}$ 		& $\displaystyle\sbra{\frac{x}{y}}$ \\[10pt]
Absolute value 			& \verb!\abs{\frac{x}{y}}!        	& $\abs{\frac{x}{y}}$ 		& $\displaystyle\abs{\frac{x}{y}}$ \\[10pt]
Set 					& \verb!\set{\frac{x}{y}, \frac{y}{z}}!        & $\set{\frac{x}{y}, \frac{y}{z}}$ 	& $\displaystyle\set{\frac{x}{y}, \frac{y}{z}}$ \\[10pt]
Floor					& \verb!\floor{\frac{x}{y}}!        	& $\floor{\frac{x}{y}}$ 		& $\displaystyle\floor{\frac{x}{y}}$ \\[10pt]
Ceiling 				& \verb!\ceil{\frac{x}{y}}!        	& $\ceil{\frac{x}{y}}$ 		& $\displaystyle\ceil{\frac{x}{y}}$ \\[10pt]
Norm					& \verb!\norm{\frac{x}{y}}!       	& $\norm{\frac{x}{y}}$ 	& $\displaystyle\norm{\frac{x}{y}}$ \\[10pt]
Inner product			& \verb!\inner{\frac{x}{y}}{\frac{y}{z}}!       	& $\inner{\frac{x}{y}}{\frac{y}{z}}$ 	& $\displaystyle\inner{\frac{x}{y}}{\frac{y}{z}}$ \\[10pt]
Cardinality 				& \verb!\card{\f[h]A}!       		& $\card{\f[h]A}$ 			& $\displaystyle\card{\f[h]A}$ \\[10pt]
\bottomrule
\end{tabular}
\end{center}

%Some examples of each size option:
%\begin{center}
%\begin{tabular}{@{}lllllll@{}}
%\toprule
%					& -1 					& 0					& 1					& 2					& 3					& 4 \\ \midrule
%
%\verb!\abs{\frac{x}{y}}!        	& $\abs[-1]{\frac{x}{y}}$ 	& $\abs[0]{\frac{x}{y}}$ 	& $\abs[1]{\frac{x}{y}}$ 	& $\abs[2]{\frac{x}{y}}$ 	& $\abs[3]{\frac{x}{y}}$ 	& $\abs[4]{\frac{x}{y}}$ \\[10pt]
%\verb!\floor{\frac{x}{y}}!        	& $\floor[-1]{\frac{x}{y}}$	& $\floor[0]{\frac{x}{y}}$	& $\floor[1]{\frac{x}{y}}$	& $\floor[2]{\frac{x}{y}}$	& $\floor[3]{\frac{x}{y}}$	& $\floor[4]{\frac{x}{y}}$ \\[10pt]
%\verb!\norm{\frac{x}{y}}!       	& $\norm[-1]{\frac{x}{y}}$	& $\norm[0]{\frac{x}{y}}$	& $\norm[1]{\frac{x}{y}}$	& $\norm[2]{\frac{x}{y}}$	& $\norm[3]{\frac{x}{y}}$	& $\norm[4]{\frac{x}{y}}$ \\[10pt]
%\bottomrule
%\end{tabular}
%\end{center}

The norm and inner product commands also have versions with a subscript argument:

\begin{center}
\begin{tabular}{@{}llll@{}}
\toprule
Description				& Example 					& Text style 				& Display style \\ \midrule
Norm	with subscript				& \verb!\normsub*{\frac{x}{y}}{2}!       	& $\normsub*{\frac{x}{y}}{2}$ 	& $\displaystyle\normsub*{\frac{x}{y}}{2}$ \\[10pt]
Inner product with subscript			& \verb!\innersub*{\frac{x}{y}}{z}{2}!       	& $\innersub*{\frac{x}{y}}{z}{2}$ 	& $\displaystyle\innersub*{\frac{x}{y}}{z}{2}$ \\[10pt]
\bottomrule
\end{tabular}
\end{center}



\section{Example Document}

TODO: a full example in basic latex versus shortex

\end{document}

