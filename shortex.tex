\documentclass{article}

\usepackage{shortex}

\title{The \texttt{ShorTeX} package}
\author{Trevor Campbell, Jonathan Huggins, and Jeff Negrea}
\date{Updated \today}


\begin{document}

\maketitle

\babs
The purpose of the ShorTeX (meta)package is to make the process of typesetting
typical mathematical documents in \LaTeX~more efficient, and the resulting
code easier to read.  It achieves this by 
(1) providing an
extensive, internally consistent, and easy to learn set of macro
shorthands, 
(2) incorporating a set of packages that are
dedicated to reducing manual coding effort, and
(3) incorporating a collection of very common / standard boilerplate packages.
\eabs

\section{Usage and package options}\label{sec:usage}


Include ShorTeX by adding \verb!\usepackage{shortex}! to the preamble of your document.
ShorTeX will include and configure many common packages for you (e.g., \texttt{graphicx, subcaption, hyperref, algorithm, algpseudocode, amsmath}, among others),
so you do not need to explicitly include and set these up yourself.
If you are writing a document that must use a specific style file (e.g., for a conference or journal) that itself
includes some of these packages, we recommend editing those style files to remove the package imports.

The ShorTeX package has a few options:
\bdesc
\item[\texttt{manualnumbering}] Do not include \texttt{autonum.sty}. This disables automatic equation numbering.
\item[\texttt{blackhypersetup}] Switch hyperlinks, citations, references, etc.~to be typeset in black font. The default is dark blue.
\edesc
\textbf{You must compile your document 4 times when using ShorTeX} to ensure that equation
numbers and references update properly.

\section{Packages included in ShorTeX}
\subsection{Internal packages}
The following packages provide features that are used internally in ShorTeX for 
macro definitions etc.
\bdesc
\item[xifthen, xstring, xspace, xargs] asdf
\edesc
\subsection{Typical packages}
We include a set of typical packages to avoid the problem of having to remember precisely
what order to import each one, which options, compatibility, etc

It is worth noting that we did not include a standard bibliography package in ShorTeX (e.g., \verb!natbib.sty!).
This is because...

\bdesc
\item[mathrsfs,dsfont,amsmath,amssymb,amsthm,bm,bbm,amsfonts,mathtools,thmtools] asdf
\item[hyperref] asdf
\item[color] asdf
\item[algorithm, algpseudocode] asdf
\item[graphicx] asdf
\edesc
\subsection{Improvement packages}
\bdesc
\item[cleveref] Typically to use a reference in \LaTeX, you have to write the name of the type of reference
yourself. For example, if you want to reference a figure, you would have to write something like:
\begin{verbatim}
In Figure \ref{fig:first}, you can see...
\end{verbatim}
Or for multiple figures, you might use:
\begin{verbatim}
Figures \ref{fig:first}, \ref{fig:second}, 
and \ref{fig:third} show that...
\end{verbatim}

The \texttt{cleveref} package simplifies this process significantly. Use the \verb!\cref! command to automatically
typeset the names of the objects you're referencing (including properly handling multiple references). The 
above two examples become
\begin{verbatim}
In \cref{fig:first}, you can see...
\end{verbatim}
and
\begin{verbatim}
\cref{fig:first,fig:second,fig:third} show that...
\end{verbatim}

This works for many different reference types (Figure, Algorithm, Equation, Table, etc),
and can be extended if needed. See the \texttt{cleveref} documentation 
at \url{https://ctan.org/pkg/cleveref?lang=en} and the homepage at \url{https://www.dr-qubit.org/cleveref.html} 
for more information.

\item[autonum] 
Typically when you typeset equations, you have to choose between 
\verb!$...$!, \verb!$$...$$!, 
\verb!\begin{align}...\end{align}!, 
\verb!\begin{aligned}...\end{aligned}!, 
\verb!\begin{equation}...\end{equation}!, 
not to mention starred versions of those environments 
and \verb!\nonumber!/\verb!\notag! commands, depending 
on whether/where you want equation numbers,
display or in-text math, etc. This leads to verbose, inconsistent code.

The \texttt{autonum} package automatically decides which equations to provide
numbers based on \textit{which equations you reference}. So when using ShorTeX,
you only need two commands for math mode: single dollar signs \verb!$...$! for
inline math, and \texttt{align} environments (redefined in ShorTeX to be
\verb!\[...\]!) for display math.\footnote{Note that there are minor
differences between how \texttt{align} and 
\texttt{equation} display equations. But after 10+ years of using \LaTeX, I have not ever
encountered a case where it mattered much. That being said, ShorTeX does not
\emph{disable} any functionality, so you can use the usual environments anywhere
you feel it is necessary.}

For example, if you create 
the following display math,
\begin{verbatim}
\[
   a+b = c \label{eq:the_equation}
\]
\end{verbatim}
then if you use the command \verb!\cref{eq:the_equation}! somewhere
in the document, that equation will automatically be assigned a number. If not, it
won't get a number. See the \texttt{autonum} package 
documentation \url{https://ctan.org/pkg/autonum?lang=en} for more information.

\item[nicefrac] asdf
\item[crossreftools] asdf
\item[multirow] asdf
\item[wrapfig] asdf
\item[caption,subcaption] asdf
\item[microtype] asdf
\item[booktabs] asdf
\item[import,subfiles] asdf
\item[url] asdf
\edesc

\section{Shortened commands and macros}

\subsection{Environments}

\LaTeX~documents often includes a lot of verbose code
related to creating environments (\verb!\begin{blah}...\end{blah}!). ShorTeX provides a set of 
shortened macros for common environments.
Note that all theorem-like environments (theorem, lemma, proposition, etc.) 
are numbered by default; unnumbered versions can be obtained by appending a \verb!u!. For example,
\verb!\bthmu...\ethmu! creates an unnumbered theorem environment, while
\verb!\blemu...\elemu! creates an unnumbered lemma environment.

\bcent
\btabr{@{}ll@{}}
\toprule
Environment & Syntax \\ \midrule
abstract & \verb!\babs...\eabs!\\ \midrule
algorithm & \verb!\balg...\ealg!\\
algorithmic & \verb!\balgc...\ealgc!\\ \midrule
table & \verb!\btab...\etab!\\
subtable & \verb!\bsubtab...\esubtab!\\
tabular & \verb!\btabr...\etabr!\\ \midrule
figure & \verb!\bfig...\efig!\\
figure* & \verb!\bfigs...\efigs!\\
subfigure & \verb!\bsubfig...\esubfig!\\ \midrule
center & \verb!\bcent...\ecent!\\ \midrule
align & \verb!\[...\]!\\ 
inline math & \verb!$...$!\\ \midrule
\multicolumn{2}{c}{\emph{Note: These are numbered theorem-like environments.}}\\
\multicolumn{2}{c}{\emph{For unnumbered, append a \texttt{u}: e.g.,} \texttt{bthmu...ethmu}.}\\
theorem & \verb!\bthm...\ethm!\\ 
lemma & \verb!\blem...\elem!\\
proposition & \verb!\bprop...\eprop!\\
corollary & \verb!\bcor...\ecor!\\
conjecture & \verb!\bconj...\econj!\\
definition & \verb!\bdef...\edef!\\
assumption & \verb!\bassump...\eassump!\\
example & \verb!\bexa...\eexa!\\
remark & \verb!\brmk...\ermk!\\
fact & \verb!\bfact...\efact!\\
exercise & \verb!\bexer...\eexer!\\
proof & \verb!\bprf...\eprf!\\
proofof & \verb!\bprfof{\cref{proof_label}}...\eprfof!\\
\etabr
\ecent


\subsection{Delimiters}

You can specify a bracket size using $*$ for \verb!\left! and \verb!\right! or one of the standard size choices (\verb!\big!, \verb!\Big!, \verb!\bigg!, \verb!\Bigg!).

\begin{center}
\begin{tabular}{@{}llll@{}}
\toprule
Description 				& Example					& Text style 				& Display style \\ \midrule
Round brackets (i.e., parentheses)	& \verb!\rbra{\frac{x}{y}}!        	& $\rbra{\frac{x}{y}}$ 		& $\displaystyle\rbra{\frac{x}{y}}$ \\[10pt]
Curly brackets 			& \verb!\cbra*{\frac{x}{y}}!    	& $\cbra*{\frac{x}{y}}$ 	& $\displaystyle\cbra*{\frac{x}{y}}$ \\[10pt]
Square brackets 			& \verb!\sbra[\bigg]{\frac{x}{y}}!        	& $\sbra[\bigg]{\frac{x}{y}}$ 	& $\displaystyle\sbra[\bigg]{\frac{x}{y}}$ \\[10pt]
\bottomrule
\end{tabular}
\end{center}

Many other bracket-like, semantic commands are also available:

\begin{center}
\begin{tabular}{@{}llll@{}}
\toprule
Description				& Example 					& Text style 				& Display style \\ \midrule
%\verb!\rbra{\frac{x}{y}}!        	& $\rbra{\frac{x}{y}}$ 		& $\displaystyle\rbra{\frac{x}{y}}$ \\[10pt]
%\verb!\cbra{\frac{x}{y}}!        	& $\cbra{\frac{x}{y}}$ 		& $\displaystyle\cbra{\frac{x}{y}}$ \\[10pt]
%\verb!\sbra{\frac{x}{y}}!        	& $\sbra{\frac{x}{y}}$ 		& $\displaystyle\sbra{\frac{x}{y}}$ \\[10pt]
Absolute value 			& \verb!\abs{\frac{x}{y}}!        	& $\abs{\frac{x}{y}}$ 		& $\displaystyle\abs{\frac{x}{y}}$ \\[10pt]
Set 					& \verb!\set{\frac{x}{y}, \frac{y}{z}}!        & $\set{\frac{x}{y}, \frac{y}{z}}$ 	& $\displaystyle\set{\frac{x}{y}, \frac{y}{z}}$ \\[10pt]
Floor					& \verb!\floor{\frac{x}{y}}!        	& $\floor{\frac{x}{y}}$ 		& $\displaystyle\floor{\frac{x}{y}}$ \\[10pt]
Ceiling 				& \verb!\ceil{\frac{x}{y}}!        	& $\ceil{\frac{x}{y}}$ 		& $\displaystyle\ceil{\frac{x}{y}}$ \\[10pt]
Norm					& \verb!\norm{\frac{x}{y}}!       	& $\norm{\frac{x}{y}}$ 	& $\displaystyle\norm{\frac{x}{y}}$ \\[10pt]
Inner product			& \verb!\inner{\frac{x}{y}}{\frac{y}{z}}!       	& $\inner{\frac{x}{y}}{\frac{y}{z}}$ 	& $\displaystyle\inner{\frac{x}{y}}{\frac{y}{z}}$ \\[10pt]
Cardinality 				& \verb!\card{\whA}!       		& $\card{\whA}$ 			& $\displaystyle\card{\whA}$ \\[10pt]
\bottomrule
\end{tabular}
\end{center}

%Some examples of each size option:
%\begin{center}
%\begin{tabular}{@{}lllllll@{}}
%\toprule
%					& -1 					& 0					& 1					& 2					& 3					& 4 \\ \midrule
%
%\verb!\abs{\frac{x}{y}}!        	& $\abs[-1]{\frac{x}{y}}$ 	& $\abs[0]{\frac{x}{y}}$ 	& $\abs[1]{\frac{x}{y}}$ 	& $\abs[2]{\frac{x}{y}}$ 	& $\abs[3]{\frac{x}{y}}$ 	& $\abs[4]{\frac{x}{y}}$ \\[10pt]
%\verb!\floor{\frac{x}{y}}!        	& $\floor[-1]{\frac{x}{y}}$	& $\floor[0]{\frac{x}{y}}$	& $\floor[1]{\frac{x}{y}}$	& $\floor[2]{\frac{x}{y}}$	& $\floor[3]{\frac{x}{y}}$	& $\floor[4]{\frac{x}{y}}$ \\[10pt]
%\verb!\norm{\frac{x}{y}}!       	& $\norm[-1]{\frac{x}{y}}$	& $\norm[0]{\frac{x}{y}}$	& $\norm[1]{\frac{x}{y}}$	& $\norm[2]{\frac{x}{y}}$	& $\norm[3]{\frac{x}{y}}$	& $\norm[4]{\frac{x}{y}}$ \\[10pt]
%\bottomrule
%\end{tabular}
%\end{center}

The norm and inner product commands also have versions with a subscript argument:

\begin{center}
\begin{tabular}{@{}llll@{}}
\toprule
Description				& Example 					& Text style 				& Display style \\ \midrule
Norm	with subscript				& \verb!\normsub*{\frac{x}{y}}{2}!       	& $\normsub*{\frac{x}{y}}{2}$ 	& $\displaystyle\normsub*{\frac{x}{y}}{2}$ \\[10pt]
Inner product with subscript			& \verb!\innersub*{\frac{x}{y}}{z}{2}!       	& $\innersub*{\frac{x}{y}}{z}{2}$ 	& $\displaystyle\innersub*{\frac{x}{y}}{z}{2}$ \\[10pt]
\bottomrule
\end{tabular}
\end{center}


\subsection{$\Lp{p}$ Spaces and Operators}

\begin{center}
\begin{tabular}{@{}llll@{}}
\toprule
Description						& Example 				& Text style 				& Display style \\ \midrule
$\Lp{p}$ space					& \verb!\Lp{2}!        		& $\Lp{2}$ 				& $\displaystyle\Lp{2}$ \\[10pt]
\begin{tabular}[c]{@{}l@{}}$\Lp{p}$ space for \\ specified measure	 \end{tabular}	& \verb!\Lpmeas{2}{\hmu}!	& $\Lpmeas{2}{\hmu}$ 		& $\displaystyle\Lpmeas{2}{\hmu}$ \\[10pt]
							& \verb!\Lpmeas[\Big]{2}{\hmu}!	& $\Lpmeas[\Big]{2}{\hmu}$ 	& $\displaystyle\Lpmeas[\Big]{2}{\hmu}$ \\[10pt]
$\Lp{p}$ norm					& \verb!\Lpnorm{\hGamma}{2}!        & $\Lpnorm{\hGamma}{2}$ 		& $\displaystyle\Lpnorm{\hGamma}{2}$ \\[10pt]
							& \verb!\Lpnorm*{\hGamma}{2}!        & $\Lpnorm*{\hGamma}{2}$ 		& $\displaystyle\Lpnorm*{\hGamma}{2}$ \\[10pt]
				& \verb!\Lpnorm*{\Gamma}{2}!        & $\Lpnorm*{\Gamma}{2}$ 		& $\displaystyle\Lpnorm*{\Gamma}{2}$ \\[10pt]
							& \verb!\left\Vert{\hGamma}\right\Vert_{2}!        & $\left\Vert{\hGamma}\right\Vert_{2}$ 		& $\displaystyle\left\Vert{\hGamma}\right\Vert_{2}$ \\[10pt]
							& \verb!\left\Vert{\Gamma}\right\Vert_{2}!        & $\left\Vert{\Gamma}\right\Vert_{2}$ 		& $\displaystyle\left\Vert{\Gamma}\right\Vert_{2}$ \\[10pt]														
\begin{tabular}[c]{@{}l@{}}$\Lp{p}$ norm for \\ specified measure	 \end{tabular}		& \verb!\Lpmeasnorm{\hGamma}{2}{\hmu}!        & $\Lpmeasnorm{\hGamma}{2}{\hmu}$ 		& $\displaystyle\Lpmeasnorm{\hGamma}{2}{\hmu}$ \\[10pt]
							& \verb!\Lpmeasnorm[\Big]{\hGamma}{2}{\hmu}!        & $\Lpmeasnorm[\Big]{\hGamma}{2}{\hmu}$ 		& $\displaystyle\Lpmeasnorm[\Big]{\hGamma}{2}{\hmu}$ \\[10pt]
$\Lp{p}$ inner product				& \verb!\Lpinner{\hGamma}{\Gamma}{2}!        & $\Lpinner{\hGamma}{\Gamma}{2}$ 		& $\displaystyle\Lpinner{\hGamma}{\Gamma}{2}$ \\[10pt]
							& \verb!\Lpinner*{\hGamma}{\Gamma}{2}!        & $\Lpinner*{\hGamma}{\Gamma}{2}$ 		& $\displaystyle\Lpinner*{\hGamma}{\Gamma}{2}$ \\[10pt]
\begin{tabular}[c]{@{}l@{}}$\Lp{p}$ inner product \\  for specified measure	 \end{tabular} & \verb!\Lpmeasinner{\hGamma}{\Gamma}{2}{\hmu}!        & $\Lpmeasinner{\hGamma}{\Gamma}{2}{\hmu}$ 		& $\displaystyle\Lpmeasinner{\hGamma}{\Gamma}{2}{\hmu}$ \\[10pt]
							& \verb!\Lpmeasinner[\big]{\hGamma}{\Gamma}{2}{\hmu}!        & $\Lpmeasinner[\big]{\hGamma}{\Gamma}{2}{\hmu}$ 		& $\displaystyle\Lpmeasinner[\big]{\hGamma}{\Gamma}{2}{\hmu}$ \\[10pt]
\bottomrule
\end{tabular}
\end{center}

\subsection{annotation commands}
\begin{tabular}{ll}
    \verb!\barA! & $\barA$ \\
    \verb!\bara! & $\bara$ \\
    \verb!\bA! & $\bA$ \\
    \verb!\bB! & $\bB$ \\
    \verb!\balpha! & $\balpha$ \\
    \verb!\bGamma! & $\bGamma$ \\
    \verb!\mcA! & $\mcA$ \\
    \verb!\hmcA! & $\hmcA$ \\
    \verb!\mfA! & $\mfA$ \\
    \verb!\mfa! & $\mfa$ \\
    \verb!\bmfA! & $\bmfA$ \\
    \verb!\bmfa! & $\bmfa$ \\
    \verb!\hA! & $\hA$ \\
    \verb!\ha! & $\ha$ \\
    \verb!\halpha! & $\halpha$ \\
    \verb!\hGamma! & $\hGamma$ \\
    \verb!\bhA! & $\bhA$ \\
    \verb!\bha! & $\bha$ \\
    \verb!\bhalpha! & $\bhalpha$ \\
    \verb!\bhGamma! & $\bhGamma$ \\
    \verb!\whA! & $\whA$ \\
    \verb!\wha! & $\wha$ \\
    \verb!\tdA! & $\tdA$ \\
    \verb!\tda! & $\tda$ \\
    \verb!\tdalpha! & $\tdalpha$ \\
    \verb!\tdGamma! & $\tdGamma$ \\
    \verb!\btdA! & $\btdA$ \\
    \verb!\btda! & $\btda$ \\
    \verb!\btdalpha! & $\btdalpha$ \\
    \verb!\btdGamma! & $\btdGamma$ \\
    \verb!\biA! & $\biA$ \\
    \verb!\bia! & $\bia$ \\
    \verb!\bhiA! & $\bhiA$ \\
    \verb!\bhia! & $\bhia$ \\
\end{tabular}


\subsection{new approach to annotation commands}
Out of the box we can get all sorts of cool fonts in math mode using \texttt{\textbackslash f[<fontcodestring>]{A}}.
For the time being I only implemented 2 fonts and 1 accent. These can be expanded once everything else is set.
We see the effect of single font codes:

\begin{tabular}{ll}
    \verb!\f[b]A! & $\f[b]A$ \\
	\verb!\f[k]A! & $\f[k]A$ \\
	% \verb!\f[c]A! & $\f[c]A$ \\
	% \verb!\f[f]A! & $\f[f]A$ \\
	% \verb!\f[y]A! & $\f[y]A$ \\
	%
	\verb!\f[h]A! & $\f[h]A$ \\
	% \verb!\f[u]A! & $\f[u]A$ \\
	% \verb!\f[o]A! & $\f[o]A$ \\
	% \verb!\f[d]A! & $\f[d]A$ \\
	% \verb!\f[t]A! & $\f[t]A$ \\
\end{tabular}

And multiple font codes:

\begin{tabular}{ll}
    \verb!\f[bh]A! & $\f[bh]A$ \\
	\verb!\f[hb]A! & $\f[hb]A$ \\
	\verb!\f[hk]A! & $\f[hk]A$ \\
	\verb!\f[kh]A! & $\f[ka]A$ \\
\end{tabular}

Note that these are expanded in the reverse of the order they appear: the font code furthest to the right is applied first. This natches the order that the corresponding commands would appear in TeX code.

We can also avoid typing \texttt{[]} for some combinations of font codes we sue frequently.
To do this, use \texttt{\textbackslash parsefontstylesstrings$\{\{$<fcstr1>$\}$,$\{$<fcstr2>$\}$,...$\}\{$<alphabet>$\}$} as demonstrated below. For ``ease of use'' we have implemented \texttt{\textbackslash upperCaseRomanLetters} and \texttt{\textbackslash lowerCaseRomanLetters}
\parsefontstylesstrings{{hb},{hk}}{ABCDEFG}. 
\parsefontstylesstrings{{hb},{hk}}{\lowerCaseRomanLetters}. 

\begin{tabular}{ll}
	\verb!\parsefontstylesstrings{{hb},{hk}}{ABCDEFG}! & ...\\
	\verb!\parsefontstylesstrings{{hb},{hk}}{\lowerCaseRomanLetters}! & ...\\
    \verb!\fhbA! & $\fhbA$ \\
	\verb!\fhkB! & $\fhkB$ \\
	\verb!\fhbx! & $\fhbx$ \\
	\verb!\fhby! & $\fhby$ \\
	\verb!\fhkz! & $\fhkz$ \\
\end{tabular}

Since \texttt{\textbackslash mathbb{<lowercaseletter>}} is defined to give weird characters, our macros do the same.


\section{Example Document}

TODO: a full example in basic latex versus shortex

\end{document}

