%%%%%%%%%%%%%%%%%%%%%%%%%%%%%%%%%%%%%%%%%%%%%%%%%%%%%%%%%%%%%%%%%%%%%%%%
%% Examples of how to use shortex.sty
%%%%%%%%%%%%%%%%%%%%%%%%%%%%%%%%%%%%%%%%%%%%%%%%%%%%%%%%%%%%%%%%%%%%%%%%


\documentclass{article}
\usepackage[autonum,colorhypersetup]{shortex}

\title{Examples of how to use \texttt{shortex.sty}}
\author{Created by Trevor Campbell, Jonathan Huggins, and Jeff Negrea}
\date{Updated \today}


\begin{document}

\maketitle


\section{Brackets and bracket-like functions}

You can specify a bracket size using $*$ for \verb!\left! and \verb!\right! or one of the standard size choices (\verb!\big!, \verb!\Big!, \verb!\bigg!, \verb!\Bigg!).

\begin{center}
\begin{tabular}{@{}llll@{}}
\toprule
Description 				& Example					& Text style 				& Display style \\ \midrule
Round brackets (i.e., parentheses)	& \verb!\rbra{\frac{x}{y}}!        	& $\rbra{\frac{x}{y}}$ 		& $\displaystyle\rbra{\frac{x}{y}}$ \\[10pt]
Curly brackets 			& \verb!\cbra*{\frac{x}{y}}!    	& $\cbra*{\frac{x}{y}}$ 	& $\displaystyle\cbra*{\frac{x}{y}}$ \\[10pt]
Square brackets 			& \verb!\sbra[\bigg]{\frac{x}{y}}!        	& $\sbra[\bigg]{\frac{x}{y}}$ 	& $\displaystyle\sbra[\bigg]{\frac{x}{y}}$ \\[10pt]
\bottomrule
\end{tabular}
\end{center}

Many other bracket-like, semantic commands are also available:

\begin{center}
\begin{tabular}{@{}llll@{}}
\toprule
Description				& Example 					& Text style 				& Display style \\ \midrule
%\verb!\rbra{\frac{x}{y}}!        	& $\rbra{\frac{x}{y}}$ 		& $\displaystyle\rbra{\frac{x}{y}}$ \\[10pt]
%\verb!\cbra{\frac{x}{y}}!        	& $\cbra{\frac{x}{y}}$ 		& $\displaystyle\cbra{\frac{x}{y}}$ \\[10pt]
%\verb!\sbra{\frac{x}{y}}!        	& $\sbra{\frac{x}{y}}$ 		& $\displaystyle\sbra{\frac{x}{y}}$ \\[10pt]
Absolute value 			& \verb!\abs{\frac{x}{y}}!        	& $\abs{\frac{x}{y}}$ 		& $\displaystyle\abs{\frac{x}{y}}$ \\[10pt]
Set 					& \verb!\set{\frac{x}{y}, \frac{y}{z}}!        & $\set{\frac{x}{y}, \frac{y}{z}}$ 	& $\displaystyle\set{\frac{x}{y}, \frac{y}{z}}$ \\[10pt]
Floor					& \verb!\floor{\frac{x}{y}}!        	& $\floor{\frac{x}{y}}$ 		& $\displaystyle\floor{\frac{x}{y}}$ \\[10pt]
Ceiling 				& \verb!\ceil{\frac{x}{y}}!        	& $\ceil{\frac{x}{y}}$ 		& $\displaystyle\ceil{\frac{x}{y}}$ \\[10pt]
Norm					& \verb!\norm{\frac{x}{y}}!       	& $\norm{\frac{x}{y}}$ 	& $\displaystyle\norm{\frac{x}{y}}$ \\[10pt]
Inner product			& \verb!\inner{\frac{x}{y}}{\frac{y}{z}}!       	& $\inner{\frac{x}{y}}{\frac{y}{z}}$ 	& $\displaystyle\inner{\frac{x}{y}}{\frac{y}{z}}$ \\[10pt]
Cardinality 				& \verb!\card{\whA}!       		& $\card{\whA}$ 			& $\displaystyle\card{\whA}$ \\[10pt]
\bottomrule
\end{tabular}
\end{center}

%Some examples of each size option:
%\begin{center}
%\begin{tabular}{@{}lllllll@{}}
%\toprule
%					& -1 					& 0					& 1					& 2					& 3					& 4 \\ \midrule
%
%\verb!\abs{\frac{x}{y}}!        	& $\abs[-1]{\frac{x}{y}}$ 	& $\abs[0]{\frac{x}{y}}$ 	& $\abs[1]{\frac{x}{y}}$ 	& $\abs[2]{\frac{x}{y}}$ 	& $\abs[3]{\frac{x}{y}}$ 	& $\abs[4]{\frac{x}{y}}$ \\[10pt]
%\verb!\floor{\frac{x}{y}}!        	& $\floor[-1]{\frac{x}{y}}$	& $\floor[0]{\frac{x}{y}}$	& $\floor[1]{\frac{x}{y}}$	& $\floor[2]{\frac{x}{y}}$	& $\floor[3]{\frac{x}{y}}$	& $\floor[4]{\frac{x}{y}}$ \\[10pt]
%\verb!\norm{\frac{x}{y}}!       	& $\norm[-1]{\frac{x}{y}}$	& $\norm[0]{\frac{x}{y}}$	& $\norm[1]{\frac{x}{y}}$	& $\norm[2]{\frac{x}{y}}$	& $\norm[3]{\frac{x}{y}}$	& $\norm[4]{\frac{x}{y}}$ \\[10pt]
%\bottomrule
%\end{tabular}
%\end{center}

The norm and inner product commands also have versions with a subscript argument:

\begin{center}
\begin{tabular}{@{}llll@{}}
\toprule
Description				& Example 					& Text style 				& Display style \\ \midrule
Norm	with subscript				& \verb!\normsub*{\frac{x}{y}}{2}!       	& $\normsub*{\frac{x}{y}}{2}$ 	& $\displaystyle\normsub*{\frac{x}{y}}{2}$ \\[10pt]
Inner product with subscript			& \verb!\innersub*{\frac{x}{y}}{z}{2}!       	& $\innersub*{\frac{x}{y}}{z}{2}$ 	& $\displaystyle\innersub*{\frac{x}{y}}{z}{2}$ \\[10pt]
\bottomrule
\end{tabular}
\end{center}


\section{$\Lp{p}$ Spaces and Operators}

\begin{center}
\begin{tabular}{@{}llll@{}}
\toprule
Description						& Example 				& Text style 				& Display style \\ \midrule
$\Lp{p}$ space					& \verb!\Lp{2}!        		& $\Lp{2}$ 				& $\displaystyle\Lp{2}$ \\[10pt]
\begin{tabular}[c]{@{}l@{}}$\Lp{p}$ space for \\ specified measure	 \end{tabular}	& \verb!\Lpmeas{2}{\hmu}!	& $\Lpmeas{2}{\hmu}$ 		& $\displaystyle\Lpmeas{2}{\hmu}$ \\[10pt]
							& \verb!\Lpmeas[\Big]{2}{\hmu}!	& $\Lpmeas[\Big]{2}{\hmu}$ 	& $\displaystyle\Lpmeas[\Big]{2}{\hmu}$ \\[10pt]
$\Lp{p}$ norm					& \verb!\Lpnorm{\hGamma}{2}!        & $\Lpnorm{\hGamma}{2}$ 		& $\displaystyle\Lpnorm{\hGamma}{2}$ \\[10pt]
							& \verb!\Lpnorm*{\hGamma}{2}!        & $\Lpnorm*{\hGamma}{2}$ 		& $\displaystyle\Lpnorm*{\hGamma}{2}$ \\[10pt]
				& \verb!\Lpnorm*{\Gamma}{2}!        & $\Lpnorm*{\Gamma}{2}$ 		& $\displaystyle\Lpnorm*{\Gamma}{2}$ \\[10pt]
							& \verb!\left\Vert{\hGamma}\right\Vert_{2}!        & $\left\Vert{\hGamma}\right\Vert_{2}$ 		& $\displaystyle\left\Vert{\hGamma}\right\Vert_{2}$ \\[10pt]
							& \verb!\left\Vert{\Gamma}\right\Vert_{2}!        & $\left\Vert{\Gamma}\right\Vert_{2}$ 		& $\displaystyle\left\Vert{\Gamma}\right\Vert_{2}$ \\[10pt]														
\begin{tabular}[c]{@{}l@{}}$\Lp{p}$ norm for \\ specified measure	 \end{tabular}		& \verb!\Lpmeasnorm{\hGamma}{2}{\hmu}!        & $\Lpmeasnorm{\hGamma}{2}{\hmu}$ 		& $\displaystyle\Lpmeasnorm{\hGamma}{2}{\hmu}$ \\[10pt]
							& \verb!\Lpmeasnorm[\Big]{\hGamma}{2}{\hmu}!        & $\Lpmeasnorm[\Big]{\hGamma}{2}{\hmu}$ 		& $\displaystyle\Lpmeasnorm[\Big]{\hGamma}{2}{\hmu}$ \\[10pt]
$\Lp{p}$ inner product				& \verb!\Lpinner{\hGamma}{\Gamma}{2}!        & $\Lpinner{\hGamma}{\Gamma}{2}$ 		& $\displaystyle\Lpinner{\hGamma}{\Gamma}{2}$ \\[10pt]
							& \verb!\Lpinner*{\hGamma}{\Gamma}{2}!        & $\Lpinner*{\hGamma}{\Gamma}{2}$ 		& $\displaystyle\Lpinner*{\hGamma}{\Gamma}{2}$ \\[10pt]
\begin{tabular}[c]{@{}l@{}}$\Lp{p}$ inner product \\  for specified measure	 \end{tabular} & \verb!\Lpmeasinner{\hGamma}{\Gamma}{2}{\hmu}!        & $\Lpmeasinner{\hGamma}{\Gamma}{2}{\hmu}$ 		& $\displaystyle\Lpmeasinner{\hGamma}{\Gamma}{2}{\hmu}$ \\[10pt]
							& \verb!\Lpmeasinner[\big]{\hGamma}{\Gamma}{2}{\hmu}!        & $\Lpmeasinner[\big]{\hGamma}{\Gamma}{2}{\hmu}$ 		& $\displaystyle\Lpmeasinner[\big]{\hGamma}{\Gamma}{2}{\hmu}$ \\[10pt]
\bottomrule
\end{tabular}
\end{center}

\section{annotation commands}
\begin{tabular}{ll}
    \verb!\barA! & $\barA$ \\
    \verb!\bara! & $\bara$ \\
    \verb!\bA! & $\bA$ \\
    \verb!\bB! & $\bB$ \\
    \verb!\balpha! & $\balpha$ \\
    \verb!\bGamma! & $\bGamma$ \\
    \verb!\mcA! & $\mcA$ \\
    \verb!\hmcA! & $\hmcA$ \\
    \verb!\mfA! & $\mfA$ \\
    \verb!\mfa! & $\mfa$ \\
    \verb!\bmfA! & $\bmfA$ \\
    \verb!\bmfa! & $\bmfa$ \\
    \verb!\hA! & $\hA$ \\
    \verb!\ha! & $\ha$ \\
    \verb!\halpha! & $\halpha$ \\
    \verb!\hGamma! & $\hGamma$ \\
    \verb!\bhA! & $\bhA$ \\
    \verb!\bha! & $\bha$ \\
    \verb!\bhalpha! & $\bhalpha$ \\
    \verb!\bhGamma! & $\bhGamma$ \\
    \verb!\whA! & $\whA$ \\
    \verb!\wha! & $\wha$ \\
    \verb!\tdA! & $\tdA$ \\
    \verb!\tda! & $\tda$ \\
    \verb!\tdalpha! & $\tdalpha$ \\
    \verb!\tdGamma! & $\tdGamma$ \\
    \verb!\btdA! & $\btdA$ \\
    \verb!\btda! & $\btda$ \\
    \verb!\btdalpha! & $\btdalpha$ \\
    \verb!\btdGamma! & $\btdGamma$ \\
    \verb!\biA! & $\biA$ \\
    \verb!\bia! & $\bia$ \\
    \verb!\bhiA! & $\bhiA$ \\
    \verb!\bhia! & $\bhia$ \\
\end{tabular}


\section{new approach to annotation commands}
Out of the box we can get all sorts of cool fonts in math mode using \texttt{\textbackslash f[<fontcodestring>]{A}}.
For the time being I only implemented 2 fonts and 1 accent. These can be expanded once everything else is set.
We see the effect of single font codes:

\begin{tabular}{ll}
    \verb!\f[b]A! & $\f[b]A$ \\
	\verb!\f[k]A! & $\f[k]A$ \\
	% \verb!\f[c]A! & $\f[c]A$ \\
	% \verb!\f[f]A! & $\f[f]A$ \\
	% \verb!\f[y]A! & $\f[y]A$ \\
	%
	\verb!\f[h]A! & $\f[h]A$ \\
	% \verb!\f[u]A! & $\f[u]A$ \\
	% \verb!\f[o]A! & $\f[o]A$ \\
	% \verb!\f[d]A! & $\f[d]A$ \\
	% \verb!\f[t]A! & $\f[t]A$ \\
\end{tabular}

And multiple font codes:

\begin{tabular}{ll}
    \verb!\f[bh]A! & $\f[bh]A$ \\
	\verb!\f[hb]A! & $\f[hb]A$ \\
	\verb!\f[hk]A! & $\f[hk]A$ \\
	\verb!\f[kh]A! & $\f[ka]A$ \\
\end{tabular}

Note that these are expanded in the reverse of the order they appear: the font code furthest to the right is applied first. This natches the order that the corresponding commands would appear in TeX code.

We can also avoid typing \texttt{[]} for some combinations of font codes we sue frequently.
To do this, use \texttt{\textbackslash parsefontstylesstrings$\{\{$<fcstr1>$\}$,$\{$<fcstr2>$\}$,...$\}\{$<alphabet>$\}$} as demonstrated below. For ``ease of use'' we have implemented \texttt{\textbackslash upperCaseRomanLetters} and \texttt{\textbackslash lowerCaseRomanLetters}
\parsefontstylesstrings{{hb},{hk}}{ABCDEFG}. 
\parsefontstylesstrings{{hb},{hk}}{\lowerCaseRomanLetters}. 

\begin{tabular}{ll}
	\verb!\parsefontstylesstrings{{hb},{hk}}{ABCDEFG}! & ...\\
	\verb!\parsefontstylesstrings{{hb},{hk}}{\lowerCaseRomanLetters}! & ...\\
    \verb!\fhbA! & $\fhbA$ \\
	\verb!\fhkB! & $\fhkB$ \\
	\verb!\fhbx! & $\fhbx$ \\
	\verb!\fhby! & $\fhby$ \\
	\verb!\fhkz! & $\fhkz$ \\
\end{tabular}

Since \texttt{\textbackslash mathbb{<lowercaseletter>}} is defined to give weird characters, our macros do the same.
\end{document}
